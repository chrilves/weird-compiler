%    Copyright (C) 2012 Christophe Calvès
%
%    This file is part of WC.
%
%    WC is free software: you can redistribute it and/or modify
%    it under the terms of the GNU General Public License as published by
%    the Free Software Foundation, either version 3 of the License, or
%    (at your option) any later version.
%
%    WC is distributed in the hope that it will be useful,
%    but WITHOUT ANY WARRANTY; without even the implied warranty of
%    MERCHANTABILITY or FITNESS FOR A PARTICULAR PURPOSE.  See the
%    GNU General Public License for more details.
%
%    You should have received a copy of the GNU General Public License
%    along with Foobar.  If not, see <http://www.gnu.org/licenses/>.
%


\starttext
\section{Installation}
 In order to compile you need:
\startitemize
\item {\tt ocaml}
\item {\tt ocamlbuild}
\item {\tt make}
\stopitemize


 You can simply type {\tt make} in the root directory to
build everything, or you can specify which part you want to
build with
\startitemize
\item {\tt make bin} to build the compiler
\item {\tt make test} to compile the examples
\item {\tt make doc} to compile this document
\item {\tt make archive} to build tar and zip source archives.
\stopitemize

\section{The Language}
 This is a minimalist functional language with strict left to right evaluation. Here is the grammar of
the language. The terminals are:
\startitemize
\item Identifiers ({\tt IDENT}) are sequences of character matching the regular expression
      {\tt [A-Za-z][A-Za-z0-9]*}.
\item Integers ({\tt INT}) are sequences of character matching the regular expression
      {\tt -?[0-9]+}.
\item Strings ({\tt STRING}) are sequences of character enclosed by double quotes.
\item Booleans are {\tt true} and {\tt false}
\stopitemize


  A program is an expression ({\tt EXPR}) composed of:
\startitemize
\item constants: {\tt INT}, booleans and {\tt STRING}.

\item variables : {\tt IDENT}.

\item functions: {\bf fun} ${\tt IDENT}_1$ \ldots\ ${\tt IDENT}_n$ {\bf ->} {\tt EXPR}. For example:
\starttyping
fun x y z -> y
\stoptyping

\item applications: {\bf(} ${\tt EXPR}_0$ ${\tt EXPR}_1$ \ldots\ ${\tt EXPR}_n$ {\bf)}. It applies the
 arguments $({\tt EXPR}_i)_{i>0}$ to ${\tt EXPR}_0$. {\bf Do not forget the brackets} or you'll be surprised!

\item let-in: {\bf let} {\tt IDENT} = ${\tt EXPR}_1$ {\bf in} ${\tt EXPR}_2$. It computes expression
${\tt EXPR}_1$, assigns its value to variable {\tt IDENT} and computes ${\tt EXPR}_2$.

\item sequences: {\bf(} ${\tt EXPR}_1$ {\bf;} \ldots\ {\bf;} ${\tt EXPR}_n$ {\bf)}. It computes
sequentialy the expressions ${\tt EXPR}_1$ to ${\tt EXPR}_n$. The last one is the value of the expression.
{\bf Do not forget the brackets} or you'll be surprised!

\item if-then-else: {\bf if} ${\tt EXPR}_1$ {\bf then} ${\tt EXPR}_2$ {\bf else} ${\tt EXPR}_3$.
If ${\tt EXPR}_1$ evaluates to {\tt true}, then ${\tt EXPR}_2$ is evaluated otherwise ${\tt EXPR}_3$
is. The value of the expression is the one of the branch being evaluated.

\stopitemize
\section{Primitives}

 Primitives are functions written directly in \TeX or in shell.


 Primitives written in \TeX must have an argumet pattern of the form {\tt \#1\ldots\#n}. They must
return a value by redefining the macro {\tt acc}. Finally, they must be placed in the file {\tt tex/extern.tex}.
For example, the addition over integers is defined in {\tt tex/extern.tex} by:
\starttyping
\def\wcplus#1#2%
{{%
\count255=#1%
\advance\count255 by #2%
\xdef\acc{\the\count255}%
}}%
\stoptyping

 Primitives written in shell must return a value by redefining the variable {\tt ACC}. They must be placed in
the file {\tt shell/extern.sh}. For example, the addition over integers is defined in {\tt shell/extern.sh} by:
\starttyping
wcplus () {
  ACC=$(($1 + $2))
}
\stoptyping

 Primitives can be used as identifiers. But they {\bf must} be declared first. A primitive is
declared by the statement: {\bf extern} {\sl primitive-name} {\sl arity} {\bf;;} at the begining
if the program. For example, {\tt wcplus} is declared by:
\starttyping
extern wcplus 2 ;;
\stoptyping

\section{Compiling and Executing}

A program is compiled to \TeX by:
\starttyping
wc.native -t -i input-file -o output-file
\stoptyping
and to POSIX shell by:\starttyping
wc.native -s -i input-file -o output-file
\stoptyping


{\tt output-file} can be run as any other shell script or \TeX file. For example, the program
{\tt first.wc}:
\starttyping
extern myprint    1 ;;
extern newline    1 ;;
extern printstats 1 ;;

let phrase = fun qui ->
 let action = fun quoi ou -> ( (myprint "The great " ) ;
                               (myprint qui          ) ;
                               (myprint " eats "     ) ;
                               (myprint quoi         ) ;
                               (myprint " in the "   ) ;
                               (myprint ou           ) ;
                               (newline 0            )
                             )
 in let pommes = action "apples"
    and poires = action "pears"
    in ( (pommes "garden." ) ;
         (pommes "garage." ) ;
         (poires "castle.") ;
         (poires "linvngroom."  )
       )
in ( ( phrase "Pierre" ) ;
     ( phrase "Paul"   ) ;
     ( phrase "Jaques" )
   )
\stoptyping

can be compied and ran to \TeX and shell by
\starttyping
wc.native -t -i first.tex -o first.tex
wc.native -s -i first.tex -o first.sh
\stoptyping

Its output is:
\starttyping
The great Pierre eats apples in the garden.
The great Pierre eats apples in the garage.
The great Pierre eats pears in the castle.
The great Pierre eats pears in the linvngroom.
The great Paul eats apples in the garden.
The great Paul eats apples in the garage.
The great Paul eats pears in the castle.
The great Paul eats pears in the linvngroom.
The great Jaques eats apples in the garden.
The great Jaques eats apples in the garage.
The great Jaques eats pears in the castle.
The great Jaques eats pears in the linvngroom.
\stoptyping

\stoptext
